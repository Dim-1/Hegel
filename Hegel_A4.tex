\documentclass[a4paper,12pt,oneside,openany]{book}%book o scrbook, da provare; twoside per libro; b5 per libro piccolo
\usepackage[T1]{fontenc}
\usepackage[utf8]{inputenc}
\usepackage[italian]{babel}
%\usepackage{layaureo}
\usepackage[eulerchapternumbers,beramono,pdfspacing]{classicthesis}
\usepackage{arsclassica}

\begin{document}
	\author{Sandro Della Maggiore}
	\title{\Huge HEGEL\\{\Large Hegel e l'idealismo assoluto}}
	\date{Marzo 2024}
	\maketitle
	\section*{Fenomenologia dello Spirito }
	
Schelling estende il concetto di intuizione alla conoscenza suprema dell’assoluto: l’assoluto è un punto di indistinzione di io e non-io e viene colto con una intuizione intellettuale, simile all’intuizione estetica. Hegel invece sostiene che l’assoluto, essendo divenire, non può essere colto da un singolo atto di intellezione, ma deve essere compreso da tutta una serie di atti, cioè mediante un ragionamento, un discorso. L’assoluto non può essere afferrato immediatamente, ma soltanto attraverso la sequenza di mediazioni in cui esso si sviluppa. La “mediazione” è presente in ogni ragionamento. Ogni ragionamento implica il partire da una premessa e lo sviluppare appunto le fila del discorso attraverso termini intermedi – di qui la parola
“mediazione” – per sostenere la propria tesi, per arrivare a dimostrare un teorema, per pervenire a una conclusione. Il ragionamento si sviluppa nel tempo e passa da un termine all’altro, è una forma di conoscenza mediata. Pertanto Hegel sostiene: non dobbiamo accettare la conoscenza intuitiva, immediata di Schelling, ma, essendo l’assoluto in divenire, essendo l’assoluto uno sviluppo, un processo, esso può essere colto soltanto mediante la discorsività, ovvero mediante il ragionamento, mediante il passaggio da un termine all’altro. A questo punto Hegel fa l’affermazione forse ancora più decisiva, che l’assoluto è un risultato: mentre per Fichte, per Schelling, l’assoluto è l’inizio, per Hegel l’assoluto è il risultato di tutto un percorso di mediazioni, è il risultato di quella sorta di enorme ragionamento di cui consiste la realtà. È come se la realtà fosse un insieme di termini ben connessi fra loro logicamente. Per cogliere l’assoluto si tratta di comprendere l’interezza della realtà in tutte le sue mediazioni, quindi non è possibile una conoscenza immediata, di tipo intuitivo, ma è necessaria una conoscenza, come dice Hegel, scientifica, che abbia la pazienza di passare da un termine all’altro, fino a giungere alla conclusione. L’assoluto è divenire, l’assoluto è soggetto, l’assoluto è risultato. Ora, il fatto che l’assoluto sia il risultato del passaggio per termini intermedi significa che bisognerà tenere presente la totalità del processo e che, se ci si fermerà semplicemente a uno dei termini intermedi, si avrà una visione falsata della realtà. Possiamo quindi aggiungere un nuovo carattere all’assoluto: l’assoluto è soggetto, è divenire, è risultato, ma è soprattutto totalità.	
	
Kant ha abbozzato una dialettica, nella Critica della ragion pura, ha contrapposto tesi e antitesi, cioè un’affermazione è uguale a se stessa e il suo contrario è uguale a se stesso: A = A, B = B. La dialettica kantiana è dicotomica, cioè consta di due termini. Con Fichte si è cominciato a introdurre un pensiero nuovo; per Fichte i termini della dialettica sono diventati tre: l’Io pone se stesso, ma l’Io nel porre se stesso pone anche il contrario di sé, il non-io; e dalla genesi del non-io scaturisce l’io empirico. Con Fichte si delinea una dialettica a tre termini, che è quella che Hegel accoglie e che estende a tutto il divenire. Per Hegel ogni cosa è identica a se stessa, ma, essendo immessa nell’ordine temporale, tende ad andare oltre se stessa, quindi è soltanto uguale a se stessa se viene vista avulsa, astratta, dal processo temporale, ma siccome invece è immessa in un processo temporale, tende a negare se stessa e a diventare diversa da quello che è, diversa da A, cioè tende a diventare B. In termini schematici questo significa che ogni elemento della realtà, che si può chiamare tesi, cioè ogni posizione, ogni cosa “A” che mi trovo posta davanti, (“tesi” nel senso etimologico da títemi, il verbo greco che significa porre) tende a trasformarsi in qualche cosa di diverso da sé, diverso da A, e quindi in non-A, cioè in B. Ogni realtà è autocontraddittoria, è identica a se stessa e tende a diventare qualche cosa di altro, di diverso da sé, quindi a ogni tesi corrisponde un’antitesi. L’antitesi è la negazione della tesi, ma non è una negazione assoluta, che distrugge la tesi: come Hegel dice, è una negazione determinata, cioè una negazione circoscritta; l’antitesi è l’annientamento di una parte della tesi per portarne in luce altri aspetti, quindi non è un processo di distruzione della tesi, bensì è un processo di superamento ( Aufhebung) della tesi stessa. Attraverso il contrasto tra quello che la cosa è e quello che la cosa tende a essere nasce un nuovo equilibrio, nasce una nuova entità, che è la sintesi di questo processo di contrapposizione. Riepiloghiamo: ogni cosa è autocontraddittoria, tende a superare l’equilibrio attualmente raggiunto e a conseguire un nuovo equilibrio; quel nuovo equilibrio è la sintesi. Nasce la triade di tesi, antitesi e sintesi.

Il divenire per autocontraddizione, la negazione determinata di ogni tesi, significano un fatto ben preciso: presa un’entità qualsiasi, che può essere un essere vivente, una figura logica, oppure un sistema sociale, essa tende a trasformarsi in qualche cosa di altro, ma non in qualsiasi altra cosa, perché, appunto, il meccanismo che la anima è di una negazione determinata; essa porta in sé qualche cosa di diverso, ma di preciso: quello che è vecchio, quello che è superato nella cosa, verrà cancellato, ma quello che c’è di fecondo, di nuovo, verrà inverato e portato a un nuovo livello. In altri termini, che cosa vuol segnalare Hegel con questo tipo di dinamica? Che il divenire è un divenire ordinato, si sviluppa secondo una logica ben precisa, che nei manuali viene spesso schematizzata con la triade di tesi, antitesi e sintesi, ma che in effetti Hegel ha messa alla prova mostrando che tutta la storia e tutto il divenire si sviluppano in questo modo.	
	
Riepiloghiamo: il vero è l’intero, il vero è la totalità, il vero è il divenire, ma il divenire si sviluppa in maniera ordinata, attraverso un meccanismo logico dialettico che la mente umana è perfettamente in grado di cogliere e di riprodurre. 	
	
Il negativo per Hegel non è negativo in assoluto, perché esso è qualche cosa di necessario per preparare uno stadio più avanzato; per Hegel tutta la storia umana e, in particolare, tutta la storia della filosofia, hanno qualche cosa da insegnare, perché attraverso le negazioni determinate hanno permesso di far salire nuovi gradini alla consapevolezza dell’uomo. Nella Fenomenologia dello spirito Hegel cerca di ripercorrere tutti questi gradini, a partire dal livello più elementare di conoscenza, cioè dalla certezza sensibile, fino a giungere al livello supremo del sapere assoluto. Quando saremo arrivati al sapere assoluto si aprirà la prima parte del sistema, cioè la logica, che considera la prima manifestazione della realtà, l’idea: l’idea è l’insieme delle strutture logiche della realtà. Ma, prima di giungere alla logica, cioè prima di descrivere la struttura del mondo ideale, che è la “tesi” della realtà, Hegel ricostruisce nella Fenomenologia dello spirito tutto il cammino che precede.
	
Nella Fenomenologia si intrecciano insomma un percorso logico e un percorso storico. È come se ci fossero due piani di discorso che si intersecano di continuo e creano qualche difficoltà di interpretazione.

Hegel iprende la grande concezione platonica: il vero si può capire soltanto in contrapposizione al falso, nasce sullo sfondo dell’errore, nasce dall’errore. In che senso riprende Platone? Perché Platone ha concepito la filosofia come dialogo e il dialogo implica posizioni che vengono superate perché sono parziali, sono erronee, sono false.

Hegel stesso dialoga con l’Illuminismo, con Kant, con Fichte e con Schelling; anzi direi con tutti i filosofi precedenti, quindi la sua posizione vera non può essere scissa dalle posizioni parzialmente erronee di Kant, di Fichte, di Schelling, e così via. Pertanto Hegel ripercorre il divenire della conoscenza umana da questo punto di vista, considerando tutto come importante perché quello che è negativo non va gettato via, ma è la premessa del positivo.	
	
Comunque sia, a questo punto diventa chiaro che quando percepisco qualche cosa la rielaboro: entriamo in una terza fase, quella dell’intelletto. Quando percepisco un oggetto lo rielaboro, e Kant su questo punto ha perfettamente ragione, e viene pienamente assorbito da Hegel: l’intelletto è uno strumento che dà forma, che unifica, che smembra, che, insomma, applica proprie forme trascendentali agli oggetti. Dalla sensazione siamo passati all’intelletto, recuperando il discorso di Kant. Ora, quando la coscienza si è pienamente sviluppata, ha raggiunto lo stadio di intelletto, che cosa avviene? L’intelletto si accorge di essere capace di manipolare gli oggetti, di dare forma agli oggetti, quindi si rende conto di se stesso.

Il soggetto, dopo essersi riversato sul mondo, arrivato all’intelletto, vede balenare un livello nuovo, cioè la conoscenza di se stesso, la consapevolezza di se stesso. L’autocoscienza è inizialmente caratterizzata da questo: sono consapevole di me, tutto il resto lo vedo come oggetto; fino a quando si tratta del bicchiere la cosa è abbastanza tranquilla, ma quando si tratta del prossimo, quando si tratta di un altro, scatta un meccanismo conflittuale. In quanto autocosciente tendo a vedere l’altro essere umano come un oggetto fra gli oggetti; naturalmente l’altro farà lo stesso nei miei confronti. Tenderò a vedere l’altro come un oggetto, o, meglio, come qualche cosa da ridurre a strumento, come il bicchiere è mio strumento. L’altro mi si presenta come un oggetto al pari degli altri oggetti, ma dall’altra parte mi si vede nello stesso modo. Si entra in una situazione di conflittualità.

Le varie autocoscienze, i vari individui, si danno a una lotta reciproca. È stato notato, giustamente, che nel passaggio dalla coscienza all’autocoscienza si passa dalla teoria alla prassi. Sensazione, percezione e intelletto sono tre forme di conoscenza degli oggetti, quando siamo arrivati all’autocoscienza, cioè alla seconda parte della Fenomenologia, entriamo nella sfera della pratica. Le autocoscienze entrano in conflitto tra loro, si fanno la guerra tra loro. Hegel riprende la teoria di Hobbes: homo homini lupus, l’uomo è lupo per l’altro uomo; l’uomo fa guerra a tutti gli altri: c’è il bellum omnium contra omnes, la guerra di tutti contro tutti. La situazione iniziale dell’autocoscienza equivale allo stato di natura di Hobbes: le autocoscienze sono in conflitto fra di loro. Che cosa succede? Che una vince e l’altra perde.

Quella che perde non viene annientata, perché serve come strumento, diventa schiavo. A questo punto Hegel cerca di far coincidere il suo ragionamento con la storia: la prima fase della storia è l’età schiavistica, l’età greco-romana sostanzialmente.

Il vincitore, l’auto-72

coscienza che nella guerra contro l’altra autocoscienza ha prevalso, riduce l’altra autocoscienza a suo strumento, a suo oggetto, come la macina, come la vanga, come l’asino, la fa diventare suo schiavo. Inizia la famosa dialettica tra il signore e il servo, signoria e servitù, che è la prima figura dell’autocoscienza (“figura” è un termine che Hegel adopera nella Fenomenologia per indicare uno stato della coscienza).

Abbiamo detto che secondo Hegel le cose non sono mai statiche, ma tendono sempre a trasformarsi. Il signore ha avuto la meglio sul servo, ma il servo deve soddisfare i bisogni del padrone, deve lavorare la terra, deve accudire gli animali, ha un contatto con la natura; il padrone perde il contatto con la natura, si limita a consumare quello che il servo ha prodotto. Allora gradualmente si viene a verificare questa situazione: il servo finisce con l’acquisire una coscienza di sé sempre maggiore, perché produce, sta a contatto con la natura, ha un sapere, un’abilità, una capacità tecnica; il signore si infiacchisce, non si confronta con le cose, ha contatto con la natura solo attraverso la mediazione del servo. A un certo punto il servo si rende conto di essere egli il vero signore, perché senza di lui il padrone muore di fame, senza di lui il padrone non può soddisfare i propri bisogni. C’è un capovolgimento dialettico. Il servo diventa padrone del proprio padrone; il padrone si trova a dipendere dal servo per il suo sostentamento, il padrone si trova ad avere un contatto con la natura soltanto indirettamente attraverso il servo, quindi paradossalmente il padrone dipende dal servo. All’inizio di questa fase dialettica il servo dipendeva dal padrone, la sua vita e la sua morte erano nelle mani del signore, ma alla fine è tutto al contrario: il servo è diventato il padrone del padrone.

Questo stadio corrisponde alla schiavitù antica e alla emancipazione dalla schiavitù; ora, il momento di emancipazione dalla schiavitù è il momento più importante della storia e anche il più delicato.
Il servo, alla fine del mondo antico, quando ha acquisito coscienza della propria superiorità, non si interessa più del mondo, nasce la filosofia stoica, che precorre elementi del Cristianesimo: il mondo non conta, sono schiavo da un punto di vista materiale, ma quello che è veramente importante è la libertà interiore, non dipendo dal mio signore, sono un uomo libero. Questo atteggiamento si intreccia con quello dello scetticismo perché per lo scettico, appunto, l’oggettività, l’esterno non conta, conta soltanto la propria consapevolezza, il proprio punto di vista sulle cose. Alla fine del mondo antico nascono le filosofie stoiche e scettiche, filosofie in cui il mondo viene negato, non è importante: non ci interessa di essere sudditi dell’imperato-re di Roma perché abbiamo la libertà dello spirito; ma soprattutto irrompe il cristianesimo che abolisce i rapporti di schiavitù e che predica che tutti gli uomini sono fratelli in Cristo. La fase iniziale del cristianesimo, che dura fino al Rinascimento, è quella della coscienza infelice. Mentre nel mondo antico, nel rapporto tra signoria e servitù, c’era uno scontro tra due autocoscienze, la personalità del signore e quella del servo, con il cristianesimo, che ha livellato gli uomini, tutti uguali perché fratelli in Cristo, la contraddizione entra nella singola coscienza, diventa un altro tipo di contraddizione. L’uomo si sente finito, si sente lontano dalla propria essenza spirituale e dall’infinito, da Dio; con un’immagine molto suggestiva Hegel dice: tutto il cristianesimo medievale converge verso le crociate, i crociati vogliono andare a vedere che c’è nel Santo Sepolcro e lo trovano vuoto. Vuole dire con questo: Dio è tornato in cielo, l’uomo è solo, con la sua finitudine, ma oramai aspira all’infinito, quindi tutta l’età cristiana medievale è l’età della coscienza infelice, vale a dire dell’uomo che si sente spossessato della sua vera natura, vive nel limitato, vive nel finito, ma ha avuto la venuta del Cristo, che gli ha lasciato intravedere il regno dei cieli, l’infinito, il sovrannaturale. L’uomo vive nel mondo naturale, ma aspira al sovrannaturale; la sua coscienza è intimamente lacerata e spezzata.

Il passaggio successivo è un passaggio a un’altra fase, completamente diversa, che è quella della ragione: è vero che la coscienza cristiana, cioè la coscienza infelice, è lacerata, ma oramai il cristiano sa che tutto quello che conta è interiore, non c’è più un elemento di antagonismo esterno come nella dialettica servo-padrone, quindi si arriva alla fase superiore, che è quella della ragione, in cui l’uomo sa 74

che oramai quello che conta sta al suo interno. La fase della ragione inizia storicamente con il Rinascimento. Con il Rinascimento si avvia il superamento della coscienza infelice, cioè del distacco tra finito e infinito. Il Rinascimento è immanentistico: Dio vive dappertutto, Dio è presente nella natura, la ragione tende a essere omogenea alla natura, a capire la natura: nasce la scienza. La ragione umana non ha più limiti, l’infinito è riconciliato col finito, l’uomo può conoscere la natura perché oramai egli ha capito che la sua ragione, che è tutto, non ha niente di esterno a sé. Inizia l’impetuoso sviluppo della scienza, con Galilei, Newton, ecc. La ragione osservativa si impadronisce sempre di più delle conoscenze del mondo. La ragione ha anche però un risvolto pratico, si manifesta anche come ragione attiva.

Nella ragione attiva Hegel identifica tre figure fondamentali. Da una parte c’è il personaggio goethiano del Faust: raggiunta la propria autonomia, la forza della ragione, l’uomo tende nella sua pratica a godersi il mondo e ad affermare se stesso nel mondo. Questo atteggiamento risulta parziale: non basta volersi affermare, l’uomo ha anche in se stesso, nei suoi rapporti pratici, una voce interiore che gli dice di amare il prossimo, partecipa anche al mondo dei sentimenti. Al momento brutale di volontà di affermazione del faustismo, succede il momento del sentimentalismo. Sia l’edonismo, cioè l’atteggiamento faustiano di volersi imporre, sia il voler essere amico di tutti, il voler avere rapporti di affettività con tutti, si rivelano però inadeguati, perché il mondo è organizzato male. Il mondo non recepisce la mia volontà di affermazione, di piacere, il faustismo, cioè la tendenza edonistica, ma non recepisce neppure la mia volontà di buoni rapporti, di rapporti fraterni, di rapporti cordiali, affettuosi, di amicizia con gli altri. Allora nasce il rigorismo della virtù, che coincide con il periodo della Rivoluzione francese: si vuole ristrutturare il mondo in modo che sia un mondo più umano. Agli atteggiamenti personali, volere il piacere, volere affermarsi, volere essere fratelli di tutti, volere buoni rapporti umani, che però sono atteggiamenti velleitari, che subiscono uno scacco, si sostituisce, alla fine dell’Età Moderna, il rigorismo della virtù. Evidentemente qui Hegel pensa a Robespierre, alla volontà di organizzare il mondo in modo che accetti la fratellanza, l’uguaglianza, la libertà; accetti i grandi valori che l’uomo ha elaborato nel corso dello sviluppo della ragione, nel corso dell’Illuminismo. A questo punto si passa all’ultima fase, che è quella dello spirito propriamente detto. Questo passaggio è complicato ed è la parte, tra l’altro, più convulsa della Fenomenologia dello spirito, quella che è stata scritta mentre sopraggiungevano le armate napo-leoniche, con Hegel divenuto ansioso di chiudere quest’opera che lo aveva tanto impegnato. Possiamo limitarci a brevi cenni perché la dottrina più matura dei momenti dello spirito Hegel la ristruttura completamente, rispetto a questo abbozzo della Fenomenologia, nell’ Enciclopedia delle scienze filosofiche.

In che cosa lo spirito si differenzia dalla ragione? La ragione è soprattutto la ragione dell’individuo che ha avuto le velleità di afferrare il piacere, di costruire rapporti di amicizia, di imporre un ordine nel mondo, ma sostanzialmente è la ragione del singolo; con lo spirito, invece, si passa a qualche cosa di più profondo. Il singolo si è scontrato con la realtà, il Faust di Goethe è caduto, è caduto il sentimentalismo, che qualcuno vuole vedere ricondotto a Rousseau, è caduto Robespierre. Lo spirito invece è una fase diversa in cui non c’è più la velleità del singolo che vuole, come Don Chisciotte, combattere contro i mulini a vento, cambiare il mondo, introdurre le sue idee nel mondo. Lo spirito non è qualche cosa di passeggero, di destinato ad essere sconfitto, a essere sorpassato: lo spirito dà luogo a creazioni permanenti. Lo spirito è il momento più alto perché è il momento in cui la comprensione dell’uomo si attaglia pienamente alla realtà, quindi dà luogo a costruzioni che rimangono, a quelle che Hegel chiama “seconda natura”. Nello spirito l’uomo è creatore.

L’uomo si trova di fronte la natura, ma crea un altro mondo, una seconda natura, che è il mondo del diritto, della famiglia, dello Stato, dell’arte, della religione, della filosofia. Lo spirito non è transeunte, non è destinato ad essere sconfitto; esso si radica nella realtà, perché corrisponde al momento più alto di comprensione dell’uomo, che veramente afferra la realtà con la sua ragione e si riesce a radicare nella realtà, riesce a impiantarvi qualche cosa di duraturo, che Hegel chiamerà nella Filosofia del diritto “seconda natura”, nel senso che è quasi una seconda creazione. Le creazioni mature dello spirito sono i grandi sistemi religiosi. I grandi sistemi religiosi cercano di cogliere l’assoluto e di organizzare popolazioni intere intorno a credenze che rimangono nei secoli, se non nei millenni, ma le religioni sono solo il penultimo stadio dello spirito, perché esse colgono l’assoluto, il divino, l’infinito, in una maniera inadeguata, ancora legata al mito, alla rappresentazione. Lo sviluppo supremo dello spirito, l’ultimo stadio della Fenomenologia, è il sapere assoluto, cioè il momento in cui l’uo-mo capisce, al di là della religione, che l’infinito, il divino, l’ideale, sono perfettamente razionali, hanno una forma razionale, e quindi devono essere capiti allo stesso livello, cioè nella forma della ragione.

La religione rimane un gradino al di sotto, l’arte un gradino ancora più sotto; perché se tutta la realtà è logica, tutta la realtà è ragione, l’assoluto è ragione, la forma adeguata per cogliere l’assoluto sarà la forma filosofica, la forma del ragionamento scientifico della filosofia.

All’assoluto razionale corrisponderà un sapere assoluto altrettanto razionale, e questo è il culmine dello sviluppo dello spirito. A questo punto la Fenomenologia dello spirito si conclude e si apre la logica, cioè il tentativo di cogliere la struttura ideale dell’assoluto.

\section*{L'idealismo logico e Enciclopedia delle scienze filosofiche in compendio}

Nel momento in cui Kant ha ridotto tutto all’elemento formale, affermando che si può conoscere solo la forma del conoscere (vale a dire intuizione, intelletto e ragione), ma non si può fare un discorso sul mondo, non si può fare un discorso sull’anima, non si può fare un discorso su Dio, cioè la metafisica è impossibile, Kant ha aperto la strada all’irrazionale nella filosofia, che poi dilaga in un certo Romanticismo. Una volta che Kant ha sostenuto che Dio non può essere oggetto di conoscenza, ha implicitamente riconosciuto che però può essere oggetto di fede (nella Critica della ragion pratica Dio è oggetto dell’esigenza dell’uomo morale), non è oggetto della ragione, ma può essere oggetto del sentimento. Quindi è come se Hegel dicesse: ‘La parte irrazionalistica della filosofia romantica è dovuta a Kant, perché Kant, avendo limitato la filosofia e la conoscenza alle forme, si è fatto sfuggire l’elemento più importante della realtà, e questo è diventato oggetto di discorsi di carattere irrazionale’.

L’assoluto, come l’ha concepito il Romanticismo, come l’ha concepito Schelling, è puntuale, cioè è qualche cosa che può essere colto con un’intuizione, con un atto di apprensione immediato: un’intuizione estetica, uno slancio di carattere mistico. Ora, il passo in avanti che Hegel compie è quello di inserire la categoria del divenire, che dal punto di vista umano è la storia, all’interno dell’assoluto. 

Se l’assoluto è una serie di mediazioni, come appunto Hegel sostiene, cioè una serie di momenti, una serie di passaggi, ciò vuol dire che è un’entità che diviene, è il risultato di uno sviluppo, per coglierlo bisogna ripercorrerne tutto lo sviluppo, la ragione è la facoltà adeguata a rispecchiarlo.

Schelling dà vita insieme all’irrazionalismo mistico e al Positivismo; se si dà infatti un esito mistico alla filosofia, dall’altra parte si ripresenta la materialità. Schelling è il padre dell’irrazionalismo dell’800 e del ’900, ma è anche il padre del positivismo, perché, per riprendere i termini di Hegel, se si limita la conoscenza al mondo finito, sensibile, materiale, allora subito si dirà che poi c’è l’assoluto, c’è Dio, c’è l’immortalità e, se non li possiamo raggiungere con la ragione, perché ci dobbiamo fermare ai sensi, li raggiungere-mo con lo slancio fideistico. L’empirismo finisce con l’aprire la strada al misticismo. Viceversa, se mi metto sulla strada mistica e dico che l’assoluto è identico a se stesso e si può cogliere solo con un rapimento estatico, mi ritrovo poi la positività delle cose, cioè le cose effettive, “poste” davanti a me, e dovrò dire che le cose effettive le conosco con i sensi. Schelling apre dunque la strada a tutte e due le forme di irrazionalismo: vorrei sottolineare che il Positivismo è anch’esso una forma di irrazionalismo.	
	
L’altra grande intuizione greca è l’altra grande affermazione di Hegel: tutto il reale è razionale significa che tutta la realtà ha una logica, non ci può essere niente, non c’è niente nella realtà naturale, animale, storica, che non sia animato da una legge, da una razionalità; tutto ciò che è reale è razionale, e quindi è comprensibile da parte dell’uomo, in quanto l’uomo è dotato di ragione, cioè è capace di rispecchiare la realtà proprio perché è il frutto più maturo dell’evoluzione naturale, è il momento in cui la natura ricomprende se stessa. In questo Hegel accoglie pienamente tutto quello che di positivo c’era in Schelling.

L’uomo con la sua ragione può conoscere tutta la realtà, niente può sfuggire alla ragione umana. È una posizione di forte fiducia nelle capacità dell’uomo.	
	
La filosofia, se è conoscenza razionale della realtà, è conoscenza del presente. La filosofia per Hegel, come dice in un’altra famosa definizione, è «il proprio tempo appreso col pensiero».

L’Illuminismo ha fallito perché pretendeva di calare ideali dalla mente dei filosofi nella realtà, invece gli ideali li partorisce la storia stessa: la storia è autocontraddittoria e genera da sé il nuovo.

Questo è l’aspetto che verrà sviluppato in particolare da Marx. «Ma la separazione della realtà dall’idea è specialmente cara all’intelletto».

La tendenza a separare reale da razionale è una delle funzioni dell’intelletto, cioè della facoltà non pienamente matura dell’uomo che tende a vedere le cose come separate, razionale da una parte e reale dall’altra: la mentalità illuministica.

«Ma la separazione della realtà dall’idea è specialmente cara all’intelletto che tiene i sogni delle sue astrazioni per alcunché di verace, ed è tutto gonfio del suo dover essere, che anche nel campo politico va predicando assai volentieri; quasi che il mondo avesse aspettato quei dettami per apprendere come deve essere e non è: ché, se poi fosse come deve essere, dove se ne andrebbe la saccenteria di quel dover essere?».

Mi pare molto efficace l’ironia di Hegel contro il dover essere: per Hegel parlare di dover essere significa contrapporre una ragione all’essere, che invece la ragione l’ha già in sé. Ciò non toglie che il reale tenda a oltrepassare se stesso. La realtà storica non è uguale a se stessa, ma è uguale a se stessa e alla realtà ulteriore che essa porta in sé, è uguale ad A, ma è anche uguale a B, che essa contiene già in embrione.

La realtà è autocontraddittoria, è in un continuo divenire che procede per tesi, antitesi e sintesi. La prima triade della realtà è: idea, natura, spirito. L’idea, dice Hegel, è come i pensieri di Dio prima che si traducessero in realtà materiale, Dio prima della creazione del mondo e dell’Incarnazione. La trama logica, la trama concettuale della realtà, cioè l’idea, viene studiata dalla logica. Questa concezione della logica porta Hegel a superare in maniera netta la logica kantiana. La logica kantiana, la logica della filosofia trascendentale, pretende di essere un discorso scientifico sulle forme della conoscenza, ma i contenuti sono esterni ad essa, invece la logica hegeliana è un discorso che non separa la forma dal contenuto: la realtà stessa al suo interno è animata da una logica, la razionalità diventa cosciente nella mente dell’uomo, ma è già presente nelle cose stesse. Per questo si dice che la logica di Hegel coincide con la metafisica: le strutture della realtà, colte dalla logica, non sono semplicemente meccanismi formali di pensiero, bensì costituiscono le strutture stesse della realtà, della realtà quale sfugge ai sensi, della realtà al di là del sensibile.

Lo studio di queste strutture logiche è nello stesso tempo studio delle strutture della realtà: la logica corrisponde alla metafisica.

In origine c’è l’idea, l’idea però è autocontraddittoria, non può rimanere semplicemente in se stessa, anch’essa è dinamica, deve negare se stessa, deve manifestarsi, invece che sul piano ideale, sul piano opposto a quello ideale, cioè sul piano materiale, deve esteriorizzarsi, deve uscire da sé ed esteriorizzarsi, cioè deve dare luogo alla natura.

Dal punto di vista logico l’idea è precedente alla cosa. L’idea, Hegel lo sottolinea con forza, viene prima della natura, si esteriorizza nella natura. Si dice che la filosofia della natura di Hegel costituisca una parte caduca del sistema. Ovviamente Hegel si è dovuto basare sugli elementi delle scienze del suo tempo, però la sua capacità di vedere la connessione tra le cose, l’applicazione del suo metodo, gli hanno permesso di precorrere intuizioni di molto successive, per esempio il fatto che lo spazio e il tempo non siano due entità separate fra di loro, come ha dimostrato Einstein, ma che siano una in funzione dell’altra, per cui la realtà è quadri-dimensionale: l’unione di spazio e tempo, che sono separati fino a Newton, fino a Kant, nelle sue lezioni sulla Filosofia della natura Hegel l’ha enunciata esplicitamente, mentre sul piano della teoria fisica è stata poi dimostrata nel Novecento.

All’interno della natura si sviluppano forme sempre più consapevoli della realtà, fino allo sboccio della razionalità, cioè della piena consapevolezza di tutto il processo di sviluppo della realtà nella razionalità umana, nell’autocoscienza umana, nello spirito. Proprio questo è il telos, il fine, il destino di tutto il divenire. Per spirito si intende la natura diventata consapevole della sua struttura ed origine ideale nell’uomo che, con la sua intelligenza, con la sua ragione, capisce la presenza dell’elemento ideale, dell’elemento razionale all’interno del reale, cioè della natura stessa. Non si tratta dello spirito nel senso metafisico: lo spirito è l’autoconsapevolezza di sé che la natura acquisisce nell’uomo, lo spirito è l’uomo razionale.

L’autoconsapevolezza che l’uomo raggiunge dell’idealità, cioè della razionalità della natura, della realtà, lo rende libero. L’uomo, consapevole di tutto il processo naturale, diventa consapevole che niente gli è estraneo, che tutto è interno alla razionalità, ma la razionalità è una sua facoltà, e quindi non c’è niente che rimanga come residuo al di fuori della sua ragione. L’uomo, nel raggiungere la piena consapevolezza razionale della realtà, ricapitola tutta la realtà: tutta la realtà è sua e l’uomo è libero. La libertà consiste in questa autoconsapevolezza. Il processo di acquisizione dell’autocoscienza è quello descritto da Hegel nella Fenomenologia dello spirito.

L’idea si è esteriorizzata nella natura; essa è poi rientrata in sé nello spirito dell’uomo. Ora, questo spirito autoconsapevole prima si manifesta come spirito isolato, come consapevolezza del singolo individuo, poi si esteriorizza oggettivandosi in una serie di creazioni dello spirito. Lo spirito dà luogo a una prima manifestazione legata all’individualità, allo spirito soggettivo, poi si proietta nelle grandi creazioni esteriori dello spirito oggettivo, infine rientra in sé nella suprema autoconsapevolezza dello spirito assoluto. Quando l’uomo ha raggiunto l’autoconsapevolezza, come l’idea si è dovuta esternare nella natura, così questa interiorità si deve negare nell’esteriorità, e dà luogo alle manifestazioni non più soggettive, bensì oggettive dello spirito.

Lo spirito oggettivo è il momento dell’interazione fra gli uomini, del consolidamento oggettivo dei loro rapporti: gli uomini non sono più isolati, singole autocoscienze, bensì vivono nella creazione di entità collettive. Il primo vincolo che gli uomini pongono tra loro è il diritto, ma il diritto è qualche cosa di limitato, in quanto pone tra gli uomini rapporti di pura esteriorità, pone norme di convivenza che vengono seguite semplicemente per il timore della coazione, per il timore della sanzione penale, per il timore del giudice, del tribunale, della prigione, cioè esclusivamente per una motivazione esteriore.

Il diritto, legato all’esteriorità, viene superato dalla morale, che è il momento in cui il rapporto con l’altro passa per una convinzione interiore; il messaggio più alto della morale è il messaggio dell’ama il prossimo tuo come te stesso: Hegel riprende pienamente la seconda formulazione kantiana dell’imperativo categorico, per cui bisogna riconoscere nell’altro un fine e non trattare mai gli altri soltanto come mezzi. Però la morale appunto è limitata dal fatto di essere qualche cosa di interiore, è una serie di relazioni fra gli uomini che sono fondate sui loro convincimenti, potremmo dire, con linguaggio cristiano, su quello che sente la loro coscienza: non è più una coazione, una costrizione esteriore che li spinge a rispettarsi, si rispettano per un sentimento che viene avvertito come proprio, come interiore. La trattazione della morale coincide in parte con le elaborazioni kantiane. Ma la morale kantiana è intenzionale: l’uomo morale vuole agire per il bene, vuole essere virtuoso, vuole realizzare la virtù, ma molto spesso, rileva Kant, il mondo è refrattario a tutto questo, perciò Kant pensa che si debba ricorrere ai postulati della ragion pratica, cioè alla speranza in un mondo in cui veramente il bene trionfa, la speranza in un Dio che veramente riesce a realizzare il bene. La morale kantiana è intenzionale. È chiaro che per Hegel questo è insoddisfacente, è come dire che si deve sognare un altro mondo in cui il bene si realizza. La morale viene superata dal mondo dell’eticità, che è un mondo in cui il bene è invece concretamente realizzato, e in cui sono congiunti l’elemento esteriore del diritto e l’elemento interiore della morale: l’eticità consiste in una serie di rapporti tra gli uomini che costituiscono vincoli oggettivi, che però vengono accettati dall’individuo come qualche cosa di profondamente radicato nella sua coscienza, qualche cosa di cui egli partecipa. Per semplificare si può fare l’esempio della famiglia: l’organizzazione della famiglia, il matrimonio è un rapporto fissato da una codificazione giuridica, come diceva Kant è anche un contratto, è un contratto però a cui si accede volontariamente e liberamente.

La famiglia è solo la prima forma di manifestazione dell’eticità: le famiglie, dice Hegel, non possono sopravvivere da sole, hanno bisogno di entrare in un rapporto con le altre, e quindi si fa strada il momento della società civile. Questo termine “civile” non implica niente di positivo: per Hegel la società civile è il mondo hobbesiano dell’ homo homini lupus, il mondo del brulicare degli interessi economici, della divisione del lavoro, delle organizzazioni umane più svariate, le quali ora cooperano per soddisfare le une i bisogni delle altre e ora si fanno la guerra per cercare di accaparrarsi profitti, spazio nella società, ecc. È il mondo delle corporazioni di lavoro, degli interessi economici, delle sette religiose, di tutte le entità organizzate a livello sovrafamiliare, però è un mondo atomizzato, frammentato.

La suprema unità invece è data dal culmine della vita etica, cioè dallo Stato. Lo Stato è concepito da Hegel in una maniera completamente diversa da tutta la tradizione precedente: tutta la tradizione precedente a partire da Hobbes sostiene una concezione contrattualistica: per il contrattualismo che è proprio di Hobbes, di Spinoza, di Locke, di Rousseau, di tutti i pensatori dell’età moderna, viene prima l’individuo e poi lo Stato. Lo Stato nasce da un contratto, cioè da un accordarsi degli individui per superare i reciproci egoismi; ci possono essere vari tipi di Stati, ma comunque il meccanismo, dall’assolutismo di Hobbes fino al liberalismo di Locke, è sempre lo stesso: l’individuo, per un fattore di convenienza, si accorda con gli altri individui e dà luogo allo Stato; quindi l’individuo rimane sempre l’elemento prioritario e lo Stato è sempre qualche cosa di derivato. In Hegel invece lo Stato è qualche cosa di assolutamente originario, logicamente precede gli individui, i quali non sono niente al di fuori dello Stato. Su questo punto leggiamo Hegel.

\begin{quotation}
	«Lo Stato è lo spirito nel quale ha luogo la prodigiosa unione dell’autonomia dell’individualità e della sostanzialità universale.

Il diritto dello Stato è quindi più alto degli alti gradi, è la libertà nella sua concreta formazione, la quale cede ancora soltanto alla suprema assoluta verità dello spirito universale».

\end{quotation}

All’interno della vita regolamentata degli uomini non c’è niente al di sopra dello Stato, l’individuo non può scavalcare lo Stato. Lo Stato, semmai può essere criticato o contestato, lo può essere soltanto da un punto di vista superiore, cioè dal punto di vista dello spirito assoluto, cioè di un’universalità ancora maggiore, ma all’interno delle organizzazioni umane nessuna organizzazione ha più universalità dello Stato. 	

\begin{quotation}
	«Lo Stato, in quanto la realtà della volontà sostanziale, che esso ha nell’autocoscienza particolare elevata alla sua universalità, è il razionale in sé e per sé».

\end{quotation}
È il razionale in sé e per sé: i pezzi del mosaico della società civile
– e insisto nel dire “mosaico”, perché è come il mosaico dei concetti dell’intelletto che sono separati gli uni dagli altri – si fanno la guerra l’uno con l’altro; gli elementi della società civile (i partiti, i sindacati, le sette religiose, le associazioni, le cooperative di produzione, ecc.) sono sempre entità parziali: credo che nessuno possa sostenere che una cooperativa o un sindacato oppure una setta religiosa contiene nella sua organizzazione un elemento di universalità, cioè di rego-lamentazione generale dei rapporti fra i suoi membri, superiore alla universalità e quindi razionalità (perché razionalità significa universalità) dello Stato. Quindi lo Stato è la suprema incarnazione della razionalità; può essere inadeguato, ci può non piacere, però storicamente costituisce la più avanzata forma di relazione fra gli uomini.

\begin{quotation}
	«Lo Stato in sé e per sé è la totalità etica, la realizzazione della libertà, ed è finalità assoluta della ragione che la libertà sia reale».
\end{quotation}

In che senso lo Stato è realizzazione della libertà? La libertà interiore per Hegel è sterile; non è veramente reale per esempio la libertà del pensiero: la libertà si deve realizzare nell’esteriorità e quindi si deve manifestare nell’organizzazione dei rapporti umani. Come si manifesta tutto questo? Nel dare universalità, cioè nel permettere la convivenza degli uomini in maniera il più possibile universale. Dal momento che universale significa razionale e che la ragione la ritrovo in me stesso, se sono sottoposto a leggi razionali sono sottoposto alla ragione che ritrovo dentro di me e non dipendo da un’entità esterna, quindi sono libero. La libertà concreta è la libertà all’interno dello Stato, non è la libertà come arbitrio (quello che io penso, i miei desideri, ecc.): la mia libertà consiste nella realizzazione oggettiva di certi rapporti su una base di universalità concreta, che è quella che lo Stato permette con le leggi. Dico universalità concreta perché certo ognuno vorrebbe la Città del sole di Campanella, la comunità ideale, ma intanto storicamente quello che si è realizzato come possibilità di interazione fra gli uomini è questo tipo di Stato. Esso è contestabile, però solo dal punto di vista della filosofia, cioè di una razionalità superiore, che deve essere capace di identificare le contraddittorietà dello Stato storicamente determinato: non mi posso contrapporre allo Stato per un motivo di carattere individuale, bensì solo per un elemento di maggiore universalità che vedo possibile sviluppare al-l’interno della compagine statale. Quindi non si può mai contestare lo Stato perché per esempio non si vogliono pagare le tasse oppure perché si pensa che sia conculcato un proprio interesse particolare; si può criticare l’assetto dello Stato solo da un punto di vista più universale di quello dello Stato.

In altri termini lo Stato, in quanto è la forma storica che ha raggiunto l’universalità dei cittadini su un territorio, comprende in sé il massimo livello di razionalità possibile. Queste affermazioni hanno poi permesso con una forzatura di portare all’affermazione che Hegel è un conservatore. Per Hegel come un uomo è sempre superiore all’animale anche se è storpio, orbo o cieco, così lo Stato è sempre superiore alla società civile, cioè ad una situazione non regolamentata dei rapporti umani. Anche uno Stato imperfetto in ogni caso è sempre più razionale che la situazione dell’ homo homini lupus, della legge del più forte che domina all’interno della società civile. Una volta che gli uomini si sono organizzati in Stati entra in gioco il rapporto tra gli Stati, cioè la storia universale. La storia ha una sua razionalità, come la natura. Per Hegel sarebbe assurdo pensare che, mentre la natura segue leggi precise che la fisica può indagare, può descrivere esattamente, la storia costituisce un altro mondo in cui non vale la razionalità. Bisogna sempre stare attenti a cercare di cogliere la razionalità presente nella storia.

Il processo storico, che ha sempre una sua razionalità, è analizzato in maniera molto originale da Hegel. Vorrei ricordare almeno due concetti. Prima di tutto quello di astuzia della ragione. C’è una razionalità nella storia che l’individuo può anche non condividere, di cui l’individuo si può anche non rendere conto: per esempio in questo momento siamo sicuramente soggetti alla legge di gravità, alla legge di inerzia, a tutta una serie di leggi fisiche, però, o perché non abbiamo studiato fisica, o perché non ce ne interessiamo in questo momento, non ce ne accorgiamo. Allo stesso modo posso agire storicamente come Napoleone Bonaparte, come Garibaldi, o come l’ultimo dei membri della comunità, senza essere consapevole di essere immesso in un flusso storico con sue leggi necessarie, oggettive e razionali ma quelle leggi necessarie, oggettive e razionali, ci sono.

Può succedere che gli intenti di protagonisti della storia divergano dalle loro azioni reali. Faccio un paio di esempi: Napoleone Bonaparte probabilmente si poneva come finalità della sua azione quella di allargare la potenza della Francia, ma quello che ha prodotto in effetti è stato la diffusione del codice civile, cioè la diffusione di un codice antifeudale e moderno di legislazione; Garibaldi probabilmente agiva nella speranza di un’Italia repubblicana, completamente unita con Roma, invece ha agito per un’Italia monarchica, che è arrivata a mezza strada rispetto a quelle che erano le sue speranze.

C’è una astuzia della ragione, vale a dire che c’è una razionalità così profonda nella storia, che i protagonisti stessi della storia e non solo gli uomini comuni possono non avvertire questa razionalità: credono di agire per certe finalità e invece la storia si prende gioco di loro, per cui portano acqua a un altro mulino, che è il mulino della razionalità profonda della storia.

Hegel nella filosofia della storia si distingue fortemente da Kant.

Kant ha scritto il progetto per una pace perpetua: egli pensa che ci possa essere una situazione in cui gli Stati, come gli individui, possono giungere a un accordo. Come gli individui si accordano nel contratto sociale e arrivano a porre fine alle contese tra di loro e si orga-nizzano nello Stato, così i singoli Stati per Kant possono raggiungere un accordo e arrivare alla pace perpetua, come suona il titolo del suo famoso libro. Questo ottimismo viene completamente negato da Hegel, che dice con sarcasmo: «Nella storia non esistono pretori». Vale a dire che nella storia non ci sono tribunali con giudici che possono mettere d’accordo i contendenti, nella storia ci sono individualità che si scontrano perennemente e di volta in volta la densità della storia si concentra in uno Stato piuttosto che in un altro; c’è uno spirito del mondo che si manifesta nei vari popoli, ma che li porta purtroppo continuamente a cozzare tra di loro, a farsi la guerra tra di loro.

Mentre Kant ha una fiducia ottimistica nella possibilità di conciliare i conflitti e arrivare a una pace perpetua, per Hegel c’è un continuo travaglio: come diceva Eraclito, Polemos è padre di tutte le cose; Polemos significa guerra e la guerra purtroppo è una realtà che nel pensiero hegeliano sembra sia qualche cosa che non può essere sormontata.

L’ultima struttura del sistema così architettonico di Hegel, è la filosofia dell’assoluto. L’assoluto, lo spirito, dopo tutte queste realizzazioni che culminano nel grande scenario della storia, deve rientrare in sé, deve ricongiungersi con se stesso, deve acquisire la totale consapevolezza di tutto il processo di sviluppo della realtà. Questo avviene in tre forme: l’arte, la religione e la filosofia.

La prima forma, l’arte, è parziale, perché nell’arte l’assoluto si rivela sotto forma sensibile, cioè sotto forma particolare e quindi sotto una forma che non è adeguata al suo contenuto, in quanto l’assoluto è razionalità, mentre nell’arte invece l’assoluto si manifesta come sensibilità, quindi non si manifesta pienamente.

L’essenza dell’arte, della poesia, consiste proprio nel particolarizzarsi dell’universale, nel divenire immagine del concettuale.

Si sostiene che Hegel, in quanto afferma che l’arte è un momento iniziale dello spirito assoluto, seguita dalla religione e poi dalla filosofia, è il filosofo della morte dell’arte. Leggiamo le parole di Hegel da cui nasce questa interpretazione:

\begin{quotation}
	«Ma come l’arte ha il suo prima nella natura e nella storia finita della vita, così ha pure un dopo, cioè un ambito che a sua volta oltrepassa il suo modo di concepire e manifestare l’assoluto».
\end{quotation}

Vale a dire che ha un suo dopo, è destinata a finire perché è inadeguata come espressione dell’assoluto.

\begin{quotation}
	«Infatti l’arte ha ancora in se stessa un limite e passa quindi a forme più alte della coscienza. Questa limitazione determina anche il posto che noi siamo soliti assegnare all’arte nella nostra vita odierna. L’arte non vale più per noi come il modo più alto in cui la verità si dà esistenza. Nell’insieme il pensiero presto si è opposto all’arte come rappresentazione sensibilizzatrice del divino. Presso gli ebrei e i maomettani per esempio, perfino presso gli stessi greci, come si può vedere nella ferma opposizione di Platone agli dei di Omero e di Esiodo».
\end{quotation}

In questo Hegel aderisce all’idea platonica per cui l’arte è qualche cosa di parzialmente negativo, che allontana dall’idea proprio per la sua materialità; si riferisce ai maomettani per i quali non è possibile rappresentare il divino in forma sensibile: l’umanità già precocemente ha capito che il divino, che poi è l’equivalente dell’infinito, dell’assoluto, non si manifesta pienamente nell’arte. Le grandi religioni già hanno avuto più o meno diffidenza per l’arte.

Hegel viene da una cultura protestante, dice che lo sviluppo del cristianesimo, il protestantesimo, ha portato a bandire le rappresentazioni sensibili; tutto il rigoglio dell’arte della Controriforma è estraneo alla Riforma: il divino non è rappresentabile nel sensibile, va colto nell’interiorità, cioè nella religione, il secondo momento dello spirito assoluto.

\begin{quotation}
	«Ma se il contenuto compiuto è compiutamente venuto a rilievo in forme artistiche, lo spirito lungimirante ritorna da questa oggettività allontanandola da sé nel suo interno. Quest’epoca è la nostra. Si può sì sperare che l’arte si innalzi e si perfezioni sempre di più, ma la sua forma ha cessato di essere il bisogno supremo dello spirito».
\end{quotation}

Hegel in qualche modo è veramente il filosofo della morte dell’arte: l’arte è destinata ad essere oltrepassata da una forma più piena di conoscenza dell’assoluto, la filosofia. Il problema è che dal punto di vista della struttura del sistema, della logica del sistema, l’arte è superata, ma come oggi esistono ancora zone del mondo in cui c’è la schiavitù, pur essendo questa stata cancellata nell’epoca moderna dal codice di Napoleone Bonaparte, dalla Rivoluzione francese (quindi logicamente è superata), così esistono ancora tentativi di cogliere l’assoluto in forma artistica, ma questi tentativi sono superati nel senso che l’umanità ha raggiunto sostanzialmente, con la filosofia idealistica, la consapevolezza che l’assoluto si può cogliere in maniera adeguata solamente nelle forme del pensiero, e appunto le forme del pensiero sono le forme della filosofia.

C’è però un ambito intermedio, quello della religione.
Nell’arte l’assoluto, l’infinito, il divino, vengono colti in qualche cosa di esterno, nella materia, nel legno, nel marmo, nei colori, nei suoni ecc.; nella religione il divino viene colto nell’interiorità, sia pur nella forma ancora inadeguata della rappresentazione. Nell’arte e nella religione c’è un’opposizione tra esteriore ed interiore, che viene superata nella filosofia.

È interiore la filosofia, ma è interiore nella forma della facoltà suprema dell’uomo, cioè la facoltà del ragionare.

La filosofia coglie l’oggettivo, cioè l’oggettivo sviluppo dello spirito, l’oggettivo sviluppo dell’assoluto, e lo coglie nella forma più interna e più alta della sua interiorità, cioè la razionalità, quindi ri-106

congiunge l’oggettivo, la razionalità presente nel mondo, l’assoluto presente in ogni momento dello sviluppo del mondo, alla propria facoltà interiore della ragione, ricongiunge oggetto e soggetto.

\begin{quote}
	«Infatti il pensiero è da un lato la soggettività più intima, più propria, e il vero pensiero, l’idea, è contemporaneamente la più oggettiva».
\end{quote}

Vale a dire che il logos è interno all’uomo, ma è anche oggettivo.

La ragione è ragione della mente umana, ma è anche ragione della realtà naturale e della realtà tout court.

\begin{quotation}
	«È contemporaneamente la più oggettiva ed effettuale universalità che può cogliersi nella sua propria forma solo nel pensiero».
\end{quotation}

L’assoluto si comprende nella filosofia, ma abbiamo detto che l’assoluto è divenire, è processo, allora anche la filosofia non potrà essere altro che processo, sviluppo, e coinciderà con la storia della filosofia. Il luogo dove l’assoluto riconosce se stesso è la storia della filosofia.
	
\section*{Lo spirito oggetivo}
	
Hegel, abbiamo detto, supera il dualismo kantiano, afferma la coincidenza di realtà e razionalità e, soprattutto, vede la realtà svilupparsi, muoversi, divenire secondo una logica precisa, la logica dialettica.

Il ritmo dialettico è presente in qualunque entità, ma anche la realtà nel suo insieme risponde a un ritmo dialettico. Nel suo complesso la realtà è idea, nel senso che essa ha una struttura ideale: ogni materia, anche apparentemente bruta, in effetti possiede una sua organizzazione, ha sempre una struttura ordinata, ha sempre una struttura “ideale”. Il fondamento della realtà è quindi l’idea. Direi quasi “per sovrabbondanza”, come nell’emanatismo di Plotino, l’idea si traduce in cosa. Il secondo momento, la natura, corrisponde in termini teologici all’Incarnazione: l’idea si fa carne e sangue, si fa materia, le strutture ideali del mondo (in questo processo che è logico, non cronologico), si concretizzano in mate ria organizzata. All’interno della materia organizzata, cioè della natura, a un certo punto nasce una forma particolarmente sviluppata, tendente alla propria consapevolezza, che è l’uomo, e si giunge al terzo e ultimo momento, lo spirito (nel linguaggio hegeliano per “spirito” si intende la consapevolezza umana). La consapevolezza umana si sviluppa anch’essa, tende a forme di conoscenza sempre più alte. In un primo momento lo spirito si presenta come lo spirito del singolo individuo. La consapevolezza del singolo individuo cresce dalla sensazione alla percezione, all’intelletto, all’autocoscienza. Quando l’uomo è diventato consapevole di sé, entra in rapporto con gli altri uomini, nasce una dinamica tra l’io e gli altri io, e si passa dallo spirito soggettivo allo spirito oggettivo. Vale a dire che gli individui si collegano agli altri individui, gli io autocoscienti si collegano alle altre autocoscienze, dando luogo a forme di organizzazione dei loro rapporti che Hegel appunto chiama spirito oggettivo.

La Filosofia del diritto tratta dello spirito oggettivo. Perché “oggettivo”? Perché la coscienza individuale nello stadio dello spirito oggettivo entra in contatto con le altre personalità, con le altre autocoscienze, dando luogo a forme di rapporto che sono oggettive, cioè non dipendono dai soggetti. Queste forme di rapporto sono il diritto, le norme morali e le costruzioni etiche. Leggi giuridiche e norme morali sono qualche cosa di oggettivo, nel senso banale che per esempio i rapporti di compravendita, i rapporti di contratto, i rapporti di successione o di matrimonio, ecc. in Italia si sviluppano secondo il diritto dello Stato italiano, ma questo diritto trascende gli individui, è oggettivo rispetto agli individui e, nella generazione prossima, con qualche variazione inevitabile dovuta alla storia, è pensabile che si svolgeranno rapporti in base alle stesse norme. O, ancora, le norme morali sono norme perenni: «ama il prossimo tuo come te stesso» non è un precetto legato a Tizio, a Caio o a Sempronio, non è qualche cosa di soggettivo. Il legame tra gli uomini che viene stabilito dalla morale è oggettivo, gli uomini lo trovano di fronte a sé, in qualche modo trascende i singoli individui. E ancor più trascende i singoli individui l’eticità, che si concretizza nelle istituzioni della famiglia, della società civile e dello Stato: esistono famiglie, ne sono esistite in passato, ne esisteranno in futuro, sono forme che in questo momento riempiamo noi come soggetti empirici attualmente viventi, ma la famiglia come istituzione vive al di fuori di noi, è oggettiva, e così la società, e così lo Stato. Per questo Hegel chiama il momento delle creazioni collettive sfera dello spirito oggettivo.

L’opera in cui Hegel analizza lo spirito oggettivo è Lineamenti di filosofia del diritto. Tra i grandissimi apporti di Hegel alla storia del pensiero è forse quello più importante: questa sfera, cioè la sfera del diritto, della moralità e dell’eticità, non è abbandonata al caso, all’arbitrio, al capriccio, all’opinione, bensì risponde a una logica, ha una razionalità interna e va considerata, al pari della natura, come qualche cosa che ha un suo logos, una sua ragione.

Quest’opera è stata pubblicata nel 1821. Hegel è dunque prudente nella Filosofia del diritto perché si tratta di un’opera a stampa controllata dalla censura, mentre è più aperto nelle lezioni, di cui soltanto negli anni scorsi si sono iniziati a pubblicare in Germania alcuni corsi, tratti dagli appunti degli studenti. Questo ha portato alcuni studiosi a sostenere che c’è un Hegel essoterico che si rivolge al pubblico e che ha scritto la Filosofia del diritto in modo che potesse passare la censura, mentre invece nella cerchia dei suoi discepoli Hegel si esprime in termini molto più radicali, è un Hegel esoterico, un Hegel segreto.
	
Per Hegel invece il pensiero non si esprime con la divergenza, non si dimostra che si sta pensando solo quando si manifesta divergenza o ostilità contro ciò che è riconosciuto, perché ciò che è riconosciuto, vale a dire la morale corrente, lo Stato come esso è attualmente organizzato, ecc., sono una sedimentazione storica, frutto della razionalità delle generazioni che ci hanno preceduto, quindi hanno in sé una loro razionalità. Qui emerge un’altra novità di Hegel: la filosofia morale, la filosofia del diritto, non dovrà dire come il mondo deve essere, non dovrà dire quali norme morali l’uomo dovrà seguire, non si tratterà di enunciare un dover essere, si tratterà di capire qual è la razionalità già presente negli Stati, nei sistemi morali, ecc.: i sistemi morali possono essere inadeguati, ma hanno un certo livello di razionalità, gli Stati possono essere inadeguati, ma hanno un certo livello di razionalità. Nelle istituzioni il passato ha sedimentato una razionalità, non si tratta di andare ad inventarla ora, mentre invece questa è l’ingenuità presun-tuosa di tutti i superficiali. Non si tratta di escogitare adesso norme morali razionali: le norme della morale corrente hanno già una loro razionalità, diversamente si scade di nuovo nella scissione kantiana e illuministica, per cui la ragione è presente solo nella mente di alcuni individui, mentre il mondo va avanti per conto suo, la realtà è una cosa e la razionalità un’altra. Hegel, contro il dualismo illuministico e kantiano, afferma che reale e razionale sono uniti: è ridicolo pretendere che io ora, all’ultimo momento, venga a dire al mondo come deve essere razionale; il mondo ha già la sua razionalità.
	
Nella Prefazione ai Lineamenti di filosofia del diritto Hegel usa per la filosofia una prima definizione: la filosofia è lo scandaglio del razionale. Lo scandaglio è lo strumento che usano i marinai per sondare i fondali. La filosofia è lo scandaglio del razionale nel senso che, al di sotto della superficie apparentemente irrazionale, caotica, degli eventi, cerca la loro razionalità profonda.
	
Consideriamo prima di tutto la famosa frase: ciò che è razionale è reale e viceversa.

Questa frase è stata molto discussa. Tra l’altro, negli appunti presi alle lezioni la frase figura in un caso al futuro e con il senso del dovere: il razionale dovrà diventare reale. Hegel stesso si è difeso dai fraintendimenti che possono farlo apparire fautore di un appiatti-mento completo, per cui tutto quello che esiste è giustificato.	

Se si legge attentamente il passo, si nota che è come se Hegel dicesse: «La realtà è variegata, è fatta di tantissime cose, è fatta di un involucro molto variopinto, è fatta di tanti elementi accidentali, io, Hegel, non chiamo “realtà” in senso pieno, in quanto solo quello che è adeguato al proprio concetto è pienamente reale, ed è razionale, ma quanto è accidentale, casuale, contingente, non lo chiamo “reale”».

D’altra parte ha affermato: «Si tratta di riconoscere nell’apparenza del temporaneo e del transitorio la sostanza», quindi si torna allo scandaglio del razionale. Se ci si ferma alla superficie e si chiama “realtà” tutto quello che si vede allora è chiaro che la formula di Hegel non funziona: essa si riferisce alla realtà profonda, cioè alla realtà essenziale.

Ora, per cogliere la realtà essenziale bisogna andare oltre la riflessione, il sentimento e altri aspetti della coscienza soggettiva. Il sentimento, sostiene Hegel, porta a dire: «Le cose vanno in un certo modo però, peccato, dovrebbero andare in un altro», il sentimento cioè tende a essere scontento della realtà per motivi puramente soggettivi.

Può sembrare strano che egli critichi anche la riflessione, ma quando parla di “riflessione” Hegel implica l’atteggiamento intellettualistico proprio dell’Illuminismo e di Kant. Per Hegel l’intelletto è qualche cosa di parziale, di negativo, perché l’intelletto è analitico, tende a distinguere, è la forma di conoscenza tipica degli illuministi e di Kant.

Kant e gli illuministi fanno leva su un intellettualismo astratto in quanto vedono il mondo composto di tante cose “tratte fuori” le une dalle altre (“astratte” significa separate) ed essi tra l’altro separano l’essere dal dover essere, quindi si creano un sovramondo ideale che poi si dovrebbe calare nella realtà, pretendono di dare consigli alla realtà su come dovrebbe essere. Questa critica è di grande importanza in quanto implica un ripensamento della Rivoluzione francese.

Come abbiamo raccontato, Hegel ha provato un entusiasmo enorme insieme con Schelling e Hölderlin, quando aveva poco più di diciott’anni, nel collegio di Tubinga, per la Rivoluzione francese. Ha poi capito però che la Rivoluzione francese ha fallito i suoi obiettivi più alti perché ha avuto una mentalità intellettualistica, ha elaborato cioè alti ideali di uguaglianza, fratellanza e libertà, ha cercato di ca-larli nella realtà, ma la realtà non li ha recepiti. Tutta l’opera di Hegel si può considerare un tentativo di fare i conti con la Rivoluzione francese in quello che aveva di grandioso e in quello che aveva di sbagliato. Hegel quindi critica la riflessione perché rifiuta l’Illuminismo, cioè la filosofia intellettualistica che, dividendo la realtà in tanti pezzi, separa anche il mondo dalla storia, separa il dover essere dall’essere: questa separazione ha portato a un naufragio: libertà, uguaglianza e fratellanza non si sono realizzate. Hegel critica il sentimento, critica l’intelletto, e paradossalmente finisce col dare ragione alla coscienza ingenua. Afferma che ogni coscienza ingenua sa che il reale è razionale. Qui per “coscienza ingenua” si intende coscienza religiosa.

Vuol dire che l’uomo religioso sa che le cose possono andare male, ma in un quadro provvidenziale, di cui momentaneamente sfuggono i contorni, l’uomo religioso, come il Manzoni, attraverso le sofferenze, i dolori, vede alla fine trionfare la Provvidenza.

L’altro grande concetto è che se ci si ferma alla superficie non si scorge la razionalità: non bisogna scambiare tutta l’apparenza per realtà, non tutte le cose che appaiono sono realtà in senso proprio. Bisogna sondare con lo scandaglio del razionale, “trovare il polso”: non qualunque parte del corpo mi rivela il ritmo cardiaco. Questo vuol dire Hegel: non qualunque cosa di per se stessa mi rivela la razionalità di un fenomeno, debbo trovare il polso, che mi dice se l’individuo sta vivendo e sta pulsando al ritmo giusto. Trovare il polso: questa è l’opera della filosofia, cioè trovare l’essenziale. Poche righe dopo ancora nella Prefazione della Filosofia del diritto, Hegel usa un’altra espressione molto enigmatica: dice che la filosofia è il trovare la rosa nella croce. Si rifà a Lutero, che aveva ideato il suo stemma in modo molto immaginoso con una rosa, una croce, una corona. Che cosa vuol dire? C’è una croce nel presente, il presente si manifesta come guerre, brutture, violenze, eppure il filosofo (come l’uomo di fede per altri aspetti) all’interno della croce, all’interno del negativo, sa trovare il positivo, sa trovare la rosa. Notate quante definizioni della filosofia dà Hegel in due, tre pagine: la filosofia è lo scandaglio del razionale, è quella che permette di trovare il polso della realtà, cioè di capirne il ritmo essenziale, è il ritrovare la rosa nella croce cioè il ritrovare quanto c’è di positivo in ciò che è apparentemente negativo.

E aggiunge nella stessa Prefazione che criticare è facile, la mentalità superficiale critica sempre, vede solo il negativo, ma il negativo è solo il momento antitetico, è solo una parte della realtà, invece bisogna considerare la realtà nella sua totalità: il vero è l’intero.

Per egel che non è ammissibile un pensiero astratto, utopistico, che vuol calare belle idee nella realtà: bisogna confrontarsi con la realtà quale essa è.
	
Nell’ultimo capoverso della Prefazione troviamo infine la famosa immagine: la filosofia è la nottola (la civetta) di Minerva che inizia il suo volo sul far del crepuscolo. La filosofia non può essere l’anticipo di un mondo che dovrà venire, non è fantascienza, non è utopia, è il proprio tempo appreso col pensiero. Secondo Hegel la filosofia sboccia sempre al momento culminante delle civiltà: Socrate e Platone sono vissuti quando cominciava la decadenza della Grecia, ed essa si iniziava a lacerare al proprio interno, non sono fioriti quando la Grecia vinceva contro i Persiani. La filosofia, secondo Hegel, sboccia nelle forme più mature quando una civiltà si sta concludendo, quando un fenomeno storico è alla fine e perciò se ne possono individuare i tratti; il filosofo non può anticipare il futuro.




	
\end{document}